Les systèmes tendent à proposer à l'utilisateur des preuves qui affirment ou réfutent un fait. Ces mêmes systèmes cherchent néanmoins absolument à proposer un score à l'utilisateur. Le verdict semble parfois être plus important que la preuve. Mais il se pourrait bien que ce soit le contraire. Si pour une personne, un fait est acquis comme vrai, il faut lui démontrer que le fait est faux \cite{cobb2013beliefs}. Trouver des preuves est plus important qu'apporter un jugement. L'argumentaire est plus important que le score.

Le fact-checking automatique fait face à plusieurs challenges importants : le TALN, la construction automatique de bases de connaissance, l'analyse de la fiabilité d'une source, etc. Sa principale limite reste le TALN, c'est une science encore imparfaite mais qui continue de progresser \cite{jozefowicz2016exploring}.

Il existe différentes approches qui ne sont pas prises en compte dans les systèmes de fact-checking. Ils se basent tous sur la vérification stricte d'un fait, d'une phrase. Il serait en effet possible de détecter les tentatives de désinformation en se basant sur un article plutôt qu'un fait. Prendre un article dans son intégralité pourrait permettre de complémentariser ces approches. Nous avons parlé par exemple de l'analyse sentimentale et la détection d'émotions dans un texte. Les fake news ont en commun un vocabulaire simple et un sentiment d'urgence ou de danger (entre autres). Nous avons vu aussi que l'analyse des médias utilisés (détection d'image retouchées, première date de publication d'une image, etc.) est indispensable. D'autre part il est nécessaire d'analyser la source, pouvoir détecter que le site émetteur d'une fake news est en fait un site satirique.

Le fact-checking en est toujours au stade expérimental. Mais de grandes institutions s'y intéressent. Parmi elles, Google avec Knowledge Vault met l'utilisation des données liées au centre de la problématique. Le web sémantique émerge toujours peu à peu mais pourrait trouver des cas d'applications de plus en plus nombreux avec l'essor de l'IA.

D'autres approches mettent en avant le machine learning et l'interrogation de source de données sur le web.

Nous avons vu que plusieurs approches étaient possibles mais qu'aucune n'était exacte. Ces approches, comme le fact-checking automatique, sont encore jeunes, elles n'ont pas atteint un stade de maturité qui permette de tirer des conclusions définitives sur le rôle qu'elles pourraient jouer dans le cadre d'un système autonome. Il faudra sûrement attendre quelques années pour voir des systèmes plus performants \cite{gravesfactsheet}.

Mais, et nous l'avons vu avec ClaimBuster, certains systèmes apportent une aide réelle au fact checker humain. Ils peuvent devenir des outils indispensables qui évolueront au fil du temps. C'est là une preuve que le fact-checking automatique même partiel est possible.

Enfin, nous avons vu les possibilités d'évolution des systèmes et différentes approches à explorer. Les possibilitées de couplage sont encore à étudier. Une autre approche à étudier serait l'utilisation des données liées pour entraîner des algorithmes. Représenter, apprendre et raisonner avec ce type de connaissances reste la prochaine étape de l'IA et de l'apprentissage automatique \cite{nickel2016review}.
